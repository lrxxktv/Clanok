% Metódy inžinierskej práce

\documentclass[10pt,twoside,slovak,a4paper]{article}

\usepackage[slovak]{babel}
%\usepackage[T1]{fontenc}
\usepackage[IL2]{fontenc} % lepšia sadzba písmena Ľ než v T1
\usepackage[utf8]{inputenc}
\usepackage{graphicx}
\usepackage{url} % príkaz \url na formátovanie URL
\usepackage{hyperref} % odkazy v texte budú aktívne (pri niektorých triedach dokumentov spôsobuje posun textu)

\usepackage{ci}
%\usepackage{times}

\pagestyle{headings}

\title{Online kurzy\thanks{Semestrálny projekt v predmete Metódy inžinierskej práce, ak. rok 2019/20, vedenie: Martin Sabo}}

\author{\\[2pt]
	{\small Slovenská technická univerzita v Bratislave}\\
	{\small Fakulta informatiky a informačných technológií}\\
	{\small \texttt{xcernanska@stuba.sk}}
	}

\date{\small 8.11.2020}



\begin{document}

\maketitle

\begin{abstract}
\ldots
\end{abstract}



\section{Abstrakt}

Motivujte čitateľa a vysvetlite, o čom píšete. Úvod sa väčšinou nedelí na časti.

Uveďte explicitne štruktúru článku. Tu je nejaký príklad.
Základný problém, ktorý bol naznačený v úvode, je podrobnejšie vysvetlený v časti~\ref{nejaka}.
Dôležité súvislosti sú uvedené v častiach~\ref{dolezita} a~\ref{dolezitejsia}.
Záverečné poznámky prináša časť~\ref{zaver}.



\section{E-learning} \label{elearning}

Pojem e-learning sa nedá jednoznačne zadefinovať. Každý si pod ním predstaví trochu inú definíciu či už z dôvodu že už má s daným pojmom nejaké skúsenosti alebo z faktoru doby, príp. veku. No všeobecne verím že si každý tento pojem zadefinuje so vzdelávaním v elektronickom prostredí. Keďže sa rozprávame o elektronickej forme vzdelávania, dalo by sa povedať že nám poskytuje prístupnejšiu formu získavania vedomostí. V tejto dobe sa využíva čoraz viac aj pri vzdelávaní na školách na zdielanie materiálov či získavanie informácií.

\subsection{Vývoj e-learningu} \label{vyvoj}

Začiatky elektronického vzdelávania začali už pred 170 rokmi v Anglicku, kde lektor posielal práce a prijímal vypracovania pomocou emailu.[1] Aj keď to nebola ešte dokonalá forma online kurzov, ich dnešná podoba sa vyvíjala ešte ďaleko predtým ako by sme predpokladali. 
Ako to celé začalo?
Prvá internetová sieť na zdieľanie študijných materiálov bola vyvinutá v roku 1960 na Illinoisskej univerzite kde si študenti mohli pozrieť učebné materiály alebo vypočuť zaznamenané prednášky.[1]
Ďalej sa internetové vzdelávanie rozvíjalo aj v podobe hier, z ktorých ako prvá sa objavila v roku 1979 pod názvom Lemonade Stand. Hra bola súčasťou softvérových balíkov spoločnosti Apple a aj keď jej koncept bol vcelku jednoduchý predstavila myšlienku učenia sa vo virtuálnej realite väčšej skupine ľudí.[1]
Rozvoj vzdelávania v elektronickom prostredí sa ďalej rozvíjal v pomerne rýchlom tempe, začali sa rozvíjať veľké univerzitné siete, knižnice a vedecké či technické siete na zdieľanie výskumov či odborných publikácií.

\subsection{Druhy e-learningu} \label{druhy}

Elektronické vzdelávanie sa v dnešnej dobre rozvinulo natoľko že ho už je potrebné rozdeľovať do viacerých skupín podľa formy odovzdávania či prijímania informácií. Vedia sa takto prispôsobiť väčšej škále pre pokrytie ich individuálnych potrieb. 
Asynchrónne online kurzy
Tieto typy kurzov sa nekonajú v reálnom čase. Študenti dostanú materiály a úlohy na odovzdanie do stanového času. Prípadné diskutovanie väčšinou prebieha pomocou diskusných fór. Študenti nie sú povinný sa v určitý čas niekde stretávať, preto tento typ kurzov je veľmi výhodný pre tých čo majú rôzne časové obmedzenia.
Synchrónne online kurzy
Tieto typy kurzov vyžadujú pripojenie na online interakciu v stanovenom čase. Účastníci spolu komunikujú prevažne pomocou videoschôdzky či zvukového alebo textového rozhovoru. Pre tieto druhy kurzov je potrebné si vyhradiť daný čas no výhodou je pripojenie sa na dialku v reálnom čase.
Hybridné kurzy
Tento typ kurzov sa dá inak nazvať ako kombinovaný. Je to spojenie klasickej osobnej výučby s doplnením o online hodiny počas osobných stretnutí. 

\subsection{Využívanie e-learningu} \label{vyuzitie}

E-learning je vzdelávací proces, využívajúci informačné a komunikačné technológie k vytváraniu kurzov, k distribúcii študijného obsahu, ku komunikácii medzi študentami a pedagógmi a k riadeniu štúdia.
Keďže forma výučby e-lerningom využíva množstvo multimediálnych prvkov ako prezentácie, obrázky, animácie, videá, zdieľané textové súbory, video či hlasové hovory, samotné zdieľané obrazovky či online testy v dnešnej dobe naberá na obrovskej popularite. Využíva sa vo firmách, školách, na vzdelávacích kurzoch, samoštúdiu a vytvára veľa možností na ďalší rozvoj vzdelávania. 

\section{Online kurzy} \label{kurzy}

Online kurz je spôsob, ako sa naučiť nové zručnosti alebo získať nové vedomosti z pohodlia domova. Môžeme ich rozdeliť na platené, alebo ponúkané zadarmo a majú veľa rôznych foriem. Na svojej najzákladnejšej úrovni majú všetci spoločné to, že učia vedomosti alebo zručnosti človeka, ktorý ich prijíma. Online kurzy sú poskytované prostredníctvom webovej stránky a je možné ich prezerať na akomkoľvek elektronickom zariadení. Môžu mať rôznu podobu vrátane vzdelávacích videí, zvukových súborov, obrázkov, pracovných listov alebo iných dokumentov. Väčšina online kurzov má aj diskusné fóra, komunitné skupiny alebo možnosti zasielania správ, čo umožňuje študentom určitý spôsob komunikácie medzi sebou alebo s učiteľom. Online výučba je obvykle spojená s vlastným tempom, čo umožňuje väčšiu flexibilitu pri absolvovaní kurzu.

\subsection{História online kurzov} \label{historia}

Prvý úplne online kurz na získanie úveru ponúkol University of Toronto v roku 1984 [60] prostredníctvom Graduate School of Education (vtedy nazývaného OISE: Ontáriový inštitút pre štúdium vzdelávania). Téma bola „Ženy a počítače vo vzdelávaní“, zaoberajúca sa rodovými otázkami a vzdelávacím výpočtom.


\section{E-learning} \label{elearning}

Z obr.~\ref{f:rozhod} je všetko jasné. 

\begin{figure*}[tbh]
\centering
%\includegraphics[scale=1.0]{diagram.pdf}

Aj text môže byť prezentovaný ako obrázok. Stane sa z neho označný plávajúci objekt. Po vytvorení diagramu zrušte znak \texttt{\%} pred príkazom \verb|\includegraphics| označte tento riadok ako komentár (tiež pomocou znaku \texttt{\%}).
\caption{Rozhodujúci argument.}
\label{f:rozhod}
\end{figure*}


\subsection{Vývoj e-learningu} \label{vyvoj}

Základným problémom je teda\ldots{} Najprv sa pozrieme na nejaké vysvetlenie (časť~\ref{ina:nejake}), a potom na ešte nejaké (časť~\ref{ina:nejake}).\footnote{Niekedy môžete potrebovať aj poznámku pod čiarou.}

Môže sa zdať, že problém vlastne nejestvuje\cite{Coplien:MPD}, ale bolo dokázané, že to tak nie je~\cite{Czarnecki:Staged, Czarnecki:Progress}. Napriek tomu, aj dnes na webe narazíme na všelijaké pochybné názory\cite{PLP-Framework}. Dôležité veci možno \emph{zdôrazniť kurzívou}.


\subsection{Nejaké vysvetlenie} \label{ina:nejake}

Niekedy treba uviesť zoznam:

\begin{itemize}
\item jedna vec
\item druhá vec
	\begin{itemize}
	\item x
	\item y
	\end{itemize}
\end{itemize}

Ten istý zoznam, len číslovaný:

\begin{enumerate}
\item jedna vec
\item druhá vec
	\begin{enumerate}
	\item x
	\item y
	\end{enumerate}
\end{enumerate}


\subsection{Ešte nejaké vysvetlenie} \label{ina:este}

\paragraph{Veľmi dôležitá poznámka.}
Niekedy je potrebné nadpisom označiť odsek. Text pokračuje hneď za nadpisom.



\section{Dôležitá časť} \label{dolezita}




\section{Ešte dôležitejšia časť} \label{dolezitejsia}




\section{Záver} \label{zaver} % prípadne iný variant názvu



%\acknowledgement{Ak niekomu chcete poďakovať\ldots}


% týmto sa generuje zoznam literatúry z obsahu súboru literatura.bib podľa toho, na čo sa v článku odkazujete
\bibliography{literatura}
\bibliographystyle{plain} % prípadne alpha, abbrv alebo hociktorý iný
\end{document}
